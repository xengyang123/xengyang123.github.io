\documentclass[10pt, letterpaper]{article}

% Packages:
\usepackage[
    ignoreheadfoot, % set margins without considering header and footer
    top=2 cm, % seperation between body and page edge from the top
    bottom=2 cm, % seperation between body and page edge from the bottom
    left=2 cm, % seperation between body and page edge from the left
    right=2 cm, % seperation between body and page edge from the right
    footskip=1.0 cm, % seperation between body and footer
    % showframe % for debugging 
]{geometry} % for adjusting page geometry
\usepackage{titlesec} % for customizing section titles
\usepackage{tabularx} % for making tables with fixed width columns
\usepackage{array} % tabularx requires this
\usepackage[dvipsnames]{xcolor} % for coloring text
\definecolor{primaryColor}{RGB}{0, 79, 144} % define primary color
\usepackage{enumitem} % for customizing lists
\usepackage{fontawesome5} % for using icons
\usepackage{amsmath} % for math
\usepackage[
    pdftitle={John Doe's CV},
    pdfauthor={John Doe},
    pdfcreator={LaTeX with RenderCV},
    colorlinks=true,
    urlcolor=primaryColor
]{hyperref} % for links, metadata and bookmarks
\usepackage[pscoord]{eso-pic} % for floating text on the page
\usepackage{calc} % for calculating lengths
\usepackage{bookmark} % for bookmarks
\usepackage{lastpage} % for getting the total number of pages
\usepackage{changepage} % for one column entries (adjustwidth environment)
\usepackage{paracol} % for two and three column entries
\usepackage{ifthen} % for conditional statements
\usepackage{needspace} % for avoiding page brake right after the section title
\usepackage{iftex} % check if engine is pdflatex, xetex or luatex

% Ensure that generate pdf is machine readable/ATS parsable:
\ifPDFTeX
    \input{glyphtounicode}
    \pdfgentounicode=1
    % \usepackage[T1]{fontenc} % this breaks sb2nov
    \usepackage[utf8]{inputenc}
    \usepackage{lmodern}
\fi



% Some settings:
\AtBeginEnvironment{adjustwidth}{\partopsep0pt} % remove space before adjustwidth environment
\pagestyle{empty} % no header or footer
\setcounter{secnumdepth}{0} % no section numbering
\setlength{\parindent}{0pt} % no indentation
\setlength{\topskip}{0pt} % no top skip
\setlength{\columnsep}{0cm} % set column seperation
\makeatletter
\let\ps@customFooterStyle\ps@plain % Copy the plain style to customFooterStyle
\patchcmd{\ps@customFooterStyle}{\thepage}{
    \color{gray}\textit{\small John Doe - Page \thepage{} of \pageref*{LastPage}}
}{}{} % replace number by desired string
\makeatother
\pagestyle{customFooterStyle}

\titleformat{\section}{\needspace{4\baselineskip}\bfseries\large}{}{0pt}{}[\vspace{1pt}\titlerule]

\titlespacing{\section}{
    % left space:
    -1pt
}{
    % top space:
    0.3 cm
}{
    % bottom space:
    0.2 cm
} % section title spacing

\renewcommand\labelitemi{$\circ$} % custom bullet points
\newenvironment{highlights}{
    \begin{itemize}[
        topsep=0.10 cm,
        parsep=0.10 cm,
        partopsep=0pt,
        itemsep=0pt,
        leftmargin=0.4 cm + 10pt
    ]
}{
    \end{itemize}
} % new environment for highlights

\newenvironment{highlightsforbulletentries}{
    \begin{itemize}[
        topsep=0.10 cm,
        parsep=0.10 cm,
        partopsep=0pt,
        itemsep=0pt,
        leftmargin=10pt
    ]
}{
    \end{itemize}
} % new environment for highlights for bullet entries


\newenvironment{onecolentry}{
    \begin{adjustwidth}{
        0.2 cm + 0.00001 cm
    }{
        0.2 cm + 0.00001 cm
    }
}{
    \end{adjustwidth}
} % new environment for one column entries

\newenvironment{twocolentry}[2][]{
    \onecolentry
    \def\secondColumn{#2}
    \setcolumnwidth{\fill, 4.5 cm}
    \begin{paracol}{2}
}{
    \switchcolumn \raggedleft \secondColumn
    \end{paracol}
    \endonecolentry
} % new environment for two column entries

\newenvironment{header}{
    \setlength{\topsep}{0pt}\par\kern\topsep\centering\linespread{1.5}
}{
    \par\kern\topsep
} % new environment for the header

\newcommand{\placelastupdatedtext}{% \placetextbox{<horizontal pos>}{<vertical pos>}{<stuff>}
  \AddToShipoutPictureFG*{% Add <stuff> to current page foreground
    \put(
        \LenToUnit{\paperwidth-2 cm-0.2 cm+0.05cm},
        \LenToUnit{\paperheight-1.0 cm}
    ){\vtop{{\null}\makebox[0pt][c]{
        \small\color{gray}\textit{Last updated in September 2024}\hspace{\widthof{Last updated in September 2024}}
    }}}%
  }%
}%

% save the original href command in a new command:
\let\hrefWithoutArrow\href

% new command for external links:
\renewcommand{\href}[2]{\hrefWithoutArrow{#1}{\ifthenelse{\equal{#2}{}}{ }{#2 }\raisebox{.15ex}{\footnotesize \faExternalLink*}}}


\begin{document}
    \newcommand{\AND}{\unskip
        \cleaders\copy\ANDbox\hskip\wd\ANDbox
        \ignorespaces
    }
    \newsavebox\ANDbox
    \sbox\ANDbox{}

    \placelastupdatedtext
    \begin{header}
        \textbf{\fontsize{24 pt}{24 pt}\selectfont Xeng Yang}

        \vspace{0.3 cm}

        \normalsize
        \mbox{{\color{black}\footnotesize\faMapMarker*}\hspace*{0.13cm}Lincoln, Nebraska}%
        \kern 0.25 cm%
        \AND%
        \kern 0.25 cm%
        \mbox{\hrefWithoutArrow{mailto:xyang33@huskers.unl.edu}{\color{black}{\footnotesize\faEnvelope[regular]}\hspace*{0.13cm}xyang33@huskers.unl.edu}}%
        \kern 0.25 cm%
        \AND%
        \kern 0.25 cm%
        \mbox{\hrefWithoutArrow{tel:651-354-5077}{\color{black}{\footnotesize\faPhone*}\hspace*{0.13cm}651-354-5077}}%
 
    \end{header}

    \vspace{0.3 cm - 0.3 cm}


    \section{Objective}



        
        \begin{onecolentry}
Seeking to enroll in a Statistics Ph.D. program, enrich my knowledge, and research about data analysis and clustering, statistical computing, machine learning, multivariate analysis, survival analysis and biostatistics. 
        \end{onecolentry}

        \vspace{0.2 cm}


    \section{Education}


        \begin{twocolentry}{
            
            
        \textit{Expect May 2029}}
            \textbf{University of Nebraska - Lincoln}

            \textit{PhD in Statistics}
            
        \end{twocolentry}

\vspace{0.10 cm}
        \begin{onecolentry}
            \begin{highlights} 
                \item \textbf{Coursework:} Statistical Method I, Mathematical Statistics, Statistical Computing Tools 
            \end{highlights}
        \end{onecolentry}

\vspace{0.30 cm}
        
        \begin{twocolentry}{
            
            
        \textit{May 2024}}
            \textbf{Minnesota State University Mankato}

            \textit{MS in Mathematics and Statistics}
        \end{twocolentry}

        \vspace{0.10 cm}
        \begin{onecolentry}
            \begin{highlights}
                \item GPA: 3.89 
                \item \textbf{Coursework:} Linear Models, Theory of Statistic, Statistical Computing 
            \end{highlights}
        \end{onecolentry}



    
    \section{Experience}



        
        \begin{twocolentry}{
        \textit{Mankato, MN}    
            
        \textit{Aug 2021 – May 2024}}
            \textbf{Graduate Teaching Assistant}
            

        \end{twocolentry}

        \vspace{0.10 cm}
        \begin{onecolentry}
        \textit{Elementary Statistics}
            \begin{highlights}
                \item Teaching sample data, visualizations for data, types of data, data collection, measurement of center and variation, basic approaches and computing probability, discrete and continuous probability, and statistical inference with one and two population parameters
                \item Making group work activities for students to engage and share ideas on how to attempt problems and produce solutions
                \item Introducing software R and assigning R labs for students to practice and understand statistical analysis
                \item Creating exams based on a diverse group of students’ abilities through quizzes and in class participation activities

\textit{Precalculus}
                \item Taught how to graph polynomial, rational, and trigonometric functions, solve system of equations and matrix, solve for coordinates, lengths, and angles for trigonometric functions
                \item Showed applications of Law of Sine and Cosine, complex numbers, and vectors
                \item Created exam reviews and Kahoot games to encourage students to work together, contribute, and learn from each other

                
            \end{highlights}
        \end{onecolentry}


        \vspace{0.2 cm}

        \begin{twocolentry}{
        \textit{Duluth, MN}    
            
        \textit{July 2019 – July 2021}}
            \textbf{Cub Food Associate}
            
        \end{twocolentry}

        \vspace{0.10 cm}
        \begin{onecolentry}
            \begin{highlights}
                \item Communicated and established teamwork with associates in the work environment
                \item Learned how to operate machine and the importance of company goals such as to serve customers with care, strategies on how to handle products, and loyalty  
            \end{highlights}
        \end{onecolentry}

 \vspace{0.2 cm}

        \begin{twocolentry}{
        \textit{Duluth, MN}    
            
        \textit{August 2017 – July 2019}}
            \textbf{Math Tutor, AmeriCorps}
        \end{twocolentry}

        \vspace{0.10 cm}
        \begin{onecolentry}
            \begin{highlights}
                \item Enriched a diverse group of students in Grade 4 and 5 on their math foundation in group activities and discussed strategies on solving word problems
                \item Learned methods to how to deliver math materials to maximize students’ learning experiences 
                \item Helped students with their confidence and test anxiety and encouraged students to persevere with positive attitude 
                
            \end{highlights}
        \end{onecolentry}




    
    \section{Projects and Research}

        \begin{twocolentry}

            \textbf{Alternative Plan Paper - }
        \end{twocolentry}

        \vspace{0.10 cm}

        \begin{onecolentry}
        \textit{Diabetes Health Indicator with Machine Learning Techniques}
            \begin{highlights}
                \item Extensively researching on chronic health disease called “diabetes” and machine learning by reading literature reviews, articles, and research papers to have clear comprehensions on how to handle a large dataset of patients who are diagnosed with or without diabetes
                \item Understanding machine learning algorithms to produce visualizations and create adequate predictive models on a large dataset
                \item Creating efficient algorithms to split the large dataset into 0.80 train data and 0.20 test data to perform statistical methods, as well as build and select adequate models 
                \item Performed k-mode and k-prototype clustering techniques to classification problem for more visualization methods since the data has both numerical and categorical variables
            \end{highlights}
        \end{onecolentry}


        \vspace{0.2 cm}

        \begin{twocolentry}

            \textbf{Edible Mushrooms Project - }
        \end{twocolentry}

        \vspace{0.10 cm}

        \begin{onecolentry}

            \begin{highlights}
                \item Researched on what characteristic features of mushrooms were edible through articles and research papers
                \item Used software R to perform k-mode and k-prototype clustering techniques and constructed scatter plots to see what size or length of its cap and stem are edible 
                \item Plotted ROC curve accuracy for logistic regression model, decision tree, random forest, and performed cross-validation to see if models were over-fitted and McFadden’s Goodness-of-Fit was used
            \end{highlights}
        \end{onecolentry}


        \vspace{0.2 cm}

        \begin{onecolentry}
        \textit{Cardiovascular Disease Project}
            \begin{highlights}
                \item Researched on what causes cardiovascular disease through articles and research papers
                \item Used software R to produce logistic model and odds ratio since the large data being a classification problem  
                \item Worked on exploratory data analysis to have visualizations and saw that the data is well-balanced
            \end{highlights}
        \end{onecolentry}


        \vspace{0.2 cm}


    
    \section{Involvement Activities}

 \begin{twocolentry}{
        \textit{Brooking, SD}    
            
        \textit{Feb. 2023 – Mar. 2023}}
            \textbf{DSU Data Analytics Competition}
            

        \end{twocolentry}

        \vspace{0.10 cm}

        \begin{onecolentry}
        \textit{South Dakota State University}
            \begin{highlights}
                \item Communicated with team to understand large data
                \item Established teamwork/collaboration to organize and clean data 
                \item Led the team to find the problems and solutions by creating visualizations, building statistical models, and testing their accuracy 
            \end{highlights}
        \end{onecolentry}


\vspace{0.2cm}


         \begin{twocolentry}{
         
            \textit{Minneapolis,MN}
            
            \textit{Feb. 2022-Apr.2022}}
            
            \textbf{Data Derby Competition }
            
        \end{twocolentry}

        \vspace{0.10 cm}

        \begin{onecolentry}
        \textit{Minnesota State I.T. Center of Excellence}
            \begin{highlights}
                \item Promoted communication to the team
                \item Assessed team members skills to be effective for the competition
                \item Worked on organizing, cleaning, and analyzing the data using statistical methods and building models 
            \end{highlights}
        \end{onecolentry}


\section{Relevant Skills}

\begin{onecolentry}
\begin{highlights}
    
\item Program skills in R
\item Program skills in SAS
\item Communication
\item Leadership
\item Teamwork/Collaboration
\item Critical Thinking
\item Perseverance

\end{highlights}

\end{onecolentry}

\end{document}